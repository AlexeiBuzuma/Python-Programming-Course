\documentclass[10pt]{beamer}
%\documentclass[10pt]{beamer}
\usetheme[
%%% option passed to the outer theme
%    progressstyle=fixedCircCnt,   % fixedCircCnt, movingCircCnt (moving is deault)
]{Feather}

%\setbeamertemplate{blocks}[rounded][shadow=true]
%\setbeamertemplate{background canvas}[vertical shading][bottom=white,top=structure.fg!25]
%\setbeamertemplate{sidebar canvas left}[horizontal shading][left=white!40!black,right=black]


\usepackage[T2A,T1]{fontenc}
\usepackage[utf8]{inputenc}
\usepackage[russian,english]{babel}

 \usepackage{helvet}
 \usepackage{graphicx}
 \usepackage{wrapfig}
 \usepackage{listings}

 \usepackage{ragged2e}  % `\justifying` text
 \usepackage{booktabs}  % Tables
 \usepackage{tabularx}
 \usepackage{tikz}      % Diagrams
 \usetikzlibrary{calc, shapes, backgrounds}
 \usepackage{amsmath, amssymb}
 \usepackage{url}       % `\url`s

 \usepackage{color}

 \definecolor{mygreen}{rgb}{0,0.6,0}
 \definecolor{mygray}{rgb}{0.5,0.5,0.5}
 \definecolor{mymauve}{rgb}{0.58,0,0.82}
 \definecolor{mybgray}{HTML}{EDECF0}

 \lstset{ %
	 backgroundcolor=\color{mybgray},   % choose the background color; you must add \usepackage{color} or \usepackage{xcolor}
      commentstyle=\color{mygreen},    % comment style
      extendedchars=true,              % lets you use non-ASCII characters; for 8-bits encodings only, does not work with UTF-8
     keywordstyle=\color{blue},       % keyword style
     language=Python,                 % the language of the code
     rulecolor=\color{black},         % if not set, the frame-color may be changed on
     stringstyle=\color{mymauve},     % string literal style
     numbers=left,     
}

 \newcommand{\chref}[2]{
 	\href{#1}{{\usebeamercolor[bg]{Feather}#2}}
 }

 \author
 {    Aleksei Buziuma \\
 	{\ttfamily buzuma\_leha@mail.ru}
 }
 
 \subject{Python language}

 \institute[]
 {
 	Belorussian State University Of Informatics and Radio-electronics.\\
 	Faculty of computer system and networks.\\
 	Electronic Computing Machines Department.\\
 }

 \date{\today}


\title[Why you need to learn Python?] % [] is optional - is placed on the bottom of the sidebar on every slide
{
	\textbf{Why do you need to learn Python?}
}

\begin{document}
	
{\1% % this is the name of the PDF file for the background
	\begin{frame}[plain,noframenumbering] 
		\titlepage
	\end{frame}
}

% Show table of content	
\begin{frame}{Content}{}
	\tableofcontents
\end{frame}
		
%-------------------------------------------------------
\section{Introduction}
%-------------------------------------------------------

%-------------------------------------------------------
\subsection{What is Python?}
%-------------------------------------------------------


% ********************** Slide 2 ***********************
\begin{frame}{Introduction}{What is Python?}
%-------------------------------------------------------
	\begin{center}
		\includegraphics[width=0.6\textwidth]{pictures/python.png}
	\end{center}
\end{frame}
% ******************************************************

% ********************** Slide 3 ***********************
\begin{frame}{Introduction}{What is Python?}
%-------------------------------------------------------
	\begin{block}{Python}
		Python is an interpreted, object-oriented, high-level programming language with dynamic semantics.  This is a general-purpose language that can be equally well developed system applications with a graphical interface, command line utilities, scientific applications, games, applications for the Web, and much more.	
	\end{block}
\end{frame}
% *****************************************************

% ********************** Slide 4 ***********************
\begin{frame}{Introduction}{What is Python?}
%-------------------------------------------------------
	\begin{center}
		\includegraphics[width=0.9\textwidth]{pictures/1.png}
	\end{center}
\end{frame}
% ******************************************************



%-------------------------------------------------------
\subsection{Dinamic vs Static}
%-------------------------------------------------------

% ********************** Slide 5 ***********************
\begin{frame}[fragile]{Introduction}{Dinamic vs Static}
%-------------------------------------------------------
\begin{center}
\begin{block}{Dinamic -- Python}
			
\begin{lstlisting}[language=Python]
>>> variable = "string"
>>> type(variable)
<class 'str'>

>>> variable = 42
>>> type(variable)
<class 'int'>

\end{lstlisting}
\end{block}
		
\pause
		
\begin{block}{Static -- C}
			
\begin{lstlisting}[language=C]
int i = 10;
i = "string";
Compile Error
\end{lstlisting}

\end{block}

\end{center}
\end{frame}
% ******************************************************

%-------------------------------------------------------
\subsection{Strong vs Weak}
%-------------------------------------------------------

% ********************** Slide 6 ***********************
\begin{frame}[fragile]{Introduction}{Strong vs Weak}
%-------------------------------------------------------
\begin{center}
		\Huge Let's talk about JavaScript!
\end{center}
\end{frame}
% ******************************************************


% ********************** Slide 6 ***********************
\begin{frame}[fragile]{Introduction}{Strong vs Weak}
%-------------------------------------------------------
\begin{center}
	
\begin{block}{Weak -- JavaScript}
	
\begin{lstlisting}
>>> [] + []  
>>>

>>> [] + {} 
[object Object]

>>> {} + []  
0

>>> {} + {}  
NaN
\end{lstlisting}


\end{block}

\end{center}

\end{frame}
% ******************************************************


% ********************** Slide 7 ***********************
\begin{frame}[fragile]{Introduction}{Strong vs Weak}
%-------------------------------------------------------
\begin{center}
		
\begin{block}{Weak -- JavaScript}
			
\begin{lstlisting}
>>> Array(10)
,,,,,,,,,

>>> Array(10).join("wat")
watwatwatwatwatwatwatwatwat

>>> Array(10).join("wat" + 1)
wat1wat1wat1wat1wat1wat1wat1wat1wat1

>>> Array(10).join("wat" - 1) + " Batman!"      
NaNNaNNaNNaNNaNNaNNaNNaNNaN Batman!
\end{lstlisting}
			
			
\end{block}
		
\end{center}
	
\end{frame}
% ******************************************************

% ********************** Slide 8 ***********************
\begin{frame}{Introduction}{WAAAAAT?}
	%-------------------------------------------------------
	\begin{center}
		\includegraphics[width=0.9\textwidth]{pictures/wat.jpg}
	\end{center}
\end{frame}
% ******************************************************


% ********************** Slide 9 ***********************
\begin{frame}[fragile]{Introduction}{Strong vs Weak}
%-------------------------------------------------------
	
\begin{center}
		
\begin{block}{Strong -- Python}
\begin{lstlisting}[language=Python]
>>> print 13 + "13"
TypeError
>>> print [] + {}
TypeError
\end{lstlisting}
\end{block}

\pause

\begin{block}{Strong -- Ruby}
\begin{lstlisting}[language=Ruby]
irb(main)> 42 + "42"
TypeError
irb(main)> [] + {}
TypeError
\end{lstlisting}
\end{block}
		
\end{center}
	
\end{frame}
% ******************************************************

% ********************** Slide 2 ***********************
\begin{frame}{Introduction}{JS}
%-------------------------------------------------------
\begin{center}
	\includegraphics[width=0.9\textwidth]{pictures/JS.jpg}
\end{center}
\end{frame}
% ******************************************************

% ********************** Slide 7 ***********************
\begin{frame}[fragile]{Introduction}{JS}
%-------------------------------------------------------
\begin{center}
		
\begin{block}{JavaScript}
			
\begin{lstlisting}
> '5' - 3
2
> '5' + 3
'53'
>'foo' + + 'foo'
'fooNaN'
>'5' + - '2'
'5-2'
> var x = 3;
>'5' + x - x
50
>'5' - x + x
5

\end{lstlisting}
			
			
\end{block}
		
\end{center}
	
\end{frame}
% ******************************************************

% ********************** Slide 2 ***********************
\begin{frame}{Introduction}{JS}
%-------------------------------------------------------
\begin{center}
	\includegraphics[width=0.6\textwidth]{pictures/wat1.jpg}
\end{center}
\end{frame}
% ******************************************************

% ********************** Slide 2 ***********************
\begin{frame}{Introduction}{JS}
%-------------------------------------------------------
\begin{center}
	\includegraphics[width=0.8\textwidth]{pictures/no_jokes.jpg}
\end{center}
\end{frame}
% ******************************************************

%-------------------------------------------------------
\subsection{History of Python}
%-------------------------------------------------------

\begin{frame}{Introduction}{History of Python}
%-------------------------------------------------------
	\begin{tikzpicture}[overlay,remember picture]
	\node[anchor=south east,xshift=-10pt,yshift=20pt]
	at (current page.south east) {
		\includegraphics[width=0.4\textwidth]{pictures/1.jpg}
	};
	\end{tikzpicture}
	
	Guido van Rossum is a Dutch computer \\ 
	programmer who is best known as the \\
	author of the Python programming \\
	language. In the Python community, \\
	Van Rossum is known as a "Benevolent \\
	Dictator For Life" (BDFL), meaning \\
	that he continues to oversee the Python\\
	development process, making decisions \\
	where necessary.He was employed by \\
	Google from 2005 until 7 December 2012,\\
	where he spent half his time developing \\
	the Python language. In January 2013, \\
	Van Rossum started working for Dropbox.
\end{frame}
%-------------------------------------------------------

%-------------------------------------------------------
\subsection{Versions}
\begin{frame}{Introduction}{Versions}
	%-------------------------------------------------------
	\begin{block}{Implementation started | December 1989}
		\begin{itemize}
			\item First appeared - 20 February 1991, 25 years ago
		\end{itemize}
	\end{block}	
		
	\begin{block}{Python 1.0 | January 1994}
		\begin{itemize}
			\item Python 1.6 - September 5, 2000
		\end{itemize}
	\end{block}	
	
	\begin{block}{Python 2.0 | October 16, 2000}
		\begin{itemize}
			\item Python 2.7 - July 3, 2010
		\end{itemize}
	\end{block}
	
	\begin{block}{Python 3.0 | December 3, 2008}
		\begin{itemize}
			\item Python 3.5 - September 13, 2015
		\end{itemize}
	\end{block}
\end{frame}
%-------------------------------------------------------


% ********************** Slide 10 ***********************
\begin{frame}{Introduction}{Pros and cons}
%-------------------------------------------------------
	\begin{itemize}
		\item Pross
		\begin{itemize}
			\item Easy to learn
			\item Cross-platform
			\item Community
			\item Open License
			\item Efficiency
			\item Simplicity
			\item Batteries
		\end{itemize}
		
		
		\item Cons
		\begin{itemize}
			\item GIL (???)
			\item Perfomance (???)
		\end{itemize}
		
	\end{itemize}	
\end{frame}
%-------------------------------------------------------


% ********************** Slide 11 ***********************
\begin{frame}{Introduction}{Stackoverflow statistic}
%-------------------------------------------------------
	\begin{center}
		\includegraphics[width=0.6\textwidth]{pictures/stackoverflow.jpg}
	\end{center}
\end{frame}
% ******************************************************

% ********************** Slide 12 **********************
\begin{frame}{Introduction}{RedMonk statistic}
%-------------------------------------------------------
	\begin{center}
		\includegraphics[width=1\textwidth]{pictures/red_monk.png}
	\end{center}
\end{frame}
% ******************************************************

%-------------------------------------------------------
\subsection{Where Python can be used?}
%-------------------------------------------------------

% ********************** Slide 13 **********************
\begin{frame}{Introduction}{Where Python can be used?}
%-------------------------------------------------------

\begin{block}{Science}
	\begin{itemize}	
		\item NumPy - a numerical Python library (a bedrock library for anything to do with matrices)
		
		\item Pandas - a library for data analysis, similar to R’s data frames or an Excel spreadsheet, built on scipy and numpy
		
		\item Scikit-learn - rapidly turning into the default machine learning library, built on scipy
		
		\item Biopython - a bioinformatics library similar to bioperl
		
		\item SymPy - a symbolic manipulation package, written in pure Python.
	\end{itemize}
	
\end{block}
\end{frame}
% ******************************************************

% ********************** Slide 14 **********************
\begin{frame}{Introduction}{Where Python can be used?}
%-------------------------------------------------------
	
	\begin{block}{Web}
		\begin{itemize}
			\item Django
			
			\item Pyramid
			
			\item Bottle
			
			\item Flask
			
			\item Tornado
		\end{itemize}
		
	\end{block}
\end{frame}
% ******************************************************

% ********************** Slide 14 **********************
\begin{frame}{Introduction}{Where Python can be used?}
	%-------------------------------------------------------
	
	\begin{block}{High Perfomance Python}
		\begin{itemize}
			\item Cyton
			
			\item Numba
			
			\item Pythran
			
			\item PyPy
			
			\item Shed Skin
		\end{itemize}
		
	\end{block}
\end{frame}
% ******************************************************

% ********************** Slide 14 **********************
\begin{frame}{Introduction}{Where Python can be used?}
%-------------------------------------------------------
	
	\begin{block}{GUI applications}
		\begin{itemize}
			\item PyQT
			
			\item PyGTK
			
			\item Kivy
			
			\item wxWidgets
	
		\end{itemize}
		
	\end{block}
\end{frame}
% ******************************************************

%-------------------------------------------------------
\subsection{IDE}
%-------------------------------------------------------


% ********************** Slide 14 **********************
\begin{frame}[fragile]{Introduction}{IDE}
%-------------------------------------------------------
	
\begin{block}{IDE}
	\begin{itemize}
		
		\pause
		\item PyCharm
		\pause
			
		\item Sublime Text
		\pause
		
		\item Atom
		\pause	
		
		\item Vim
		\pause
		
		\item Jupyter
		\pause
		
		\item ipython, bpython
	\end{itemize}
\end{block}	
	
\end{frame}
%-------------------------------------------------------

%-------------------------------------------------------
\subsection{Jobs}
%-------------------------------------------------------

% ********************** Slide 11 ***********************
\begin{frame}{Introduction}{tut.by jobs}
%-------------------------------------------------------
\begin{center}
\includegraphics[width=1.1\textwidth]{pictures/minsk_job_tutby.png}
\end{center}
\end{frame}
% ******************************************************

%-------------------------------------------------------
\subsection{Indentation is very important!}
%-------------------------------------------------------



% ********************** Slide 14 **********************
\begin{frame}[fragile]{Introduction}{Indentation is very important!}
%-------------------------------------------------------
	
\begin{block}{Indentation}
	
\begin{lstlisting}[language=Python, tabsize=4, showspaces=true, showtabs=true]
if True:
    print("True")
else:
    print("False")

if True:
	print("True")
    print("Error")

if True:
    print("True")
  print("Error")
\end{lstlisting}

\end{block}	
	
\end{frame}


%-------------------------------------------------------
\begin{frame}{Introduction}{Questions?}
%-------------------------------------------------------
	\large Questions?
	\begin{center}
		
		\begin{block}{Contacts}
			\begin{itemize}
				\item Email:    buzuma.leha@gmail.com
				\item Facebook: aleksei.buzuma
			\end{itemize}
		\end{block}
		
	\end{center}
\end{frame}

\end{document}